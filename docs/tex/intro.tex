\section{Introduction}
\label{sec:intro}

SASSIFI can quantify how architecture-level errors propagate to the program
output level through architecture-level error injections experiments. It can be
used in two modes: (1) Inject errors in the outputs of the instructions. This
is useful if you want to find out what can happen if a low-level soft-error
manifests at the architecture-level. See the Resilience case study in the SASSI
ISCA 2015 paper~\cite{SASSI_ISCA} and SASSIFI SELSE 2015
presentation~\cite{SASSIFI_SELSE} for more details. We call this mode of
operation as {\it D-mode} because the injections are dependent on the
instruction. (2) Inject bit-flips in the Register File (RF), randomly spread
across time and space (among allocated registers). The purpose of such
injections is to compute Architectural Vulnerability Factor (AVF) of the RF.
The results would tell us the importance of using ECC/parity on the RF.  We
call this mode {\it I-mode} because the register selected for injection is
independent of the instruction executing at the time of injection. 

\section{Where can SASSIFI inject errors?}
\label{sec:where}

For the D-mode (instruction output-level injections), SASSIFI can inject
errors in the outputs of randomly selected instructions. SASSIFI allows us to
select different types of instructions to study how error in them will
propagate to the application output. As of now (7/22/2016), SASSIFI supports
selecting the following instruction groups. 
\begin{itemize}
\item Instructions that write to general purpose registers (GPR) 
\item Instructions that write to condition code (CC) 
\item Instructions that write to a predicate register (PR) 
\item Store instruction (Store value) 
\item Integer add and multiply instructions (IADD-IMAD-OP)
\item Single and double precision floating point add and multiply instructions (FADD-FMUL-OP)
\item Integer fused multiply and add (MAD) instructions (MAD-OP)
\item Single and double precision floating point fused multiply and add (FMA) instructions  (FMA-OP)
\item Instructions that compare source registers and set a predicate register (SETP-OP)
\item Loads from shared memory (LDS-OP)
\item Load instructions, excluding LDS instructions (LD-OP)
\end{itemize}

SASSIFI can be extended to include custom instruction groups. Follow
instructions in Section~\ref{sec:new_inst_group} to create new instruction
groups. Details about the current instruction grouping, i.e., which SASS
instructions are included in different groups, can be found in
\$SASSIFI\_HOME/err\_injector/error\_injector.h.


For the I-mode (injections to measure RF AVF), SASSIFI selects a dynamic
instruction randomly from a program and injects an error in a randomly selected
register among the allocated registers. The NVIDIA compiler specifies the
maximum number of registers allocated per thread (using {\it -Xptxas -v}
option) and this mode randomly selects a register from that pool for injection.
SASSIFI quantifies the masking, DUE, and SDC rates from error injections in
allocated registers.  These DUE/SDC rates can be further derated by the average
fraction of physical registers that are unallocated on a target GPU to obtain
AVF of the RF.  The average fraction of unallocated registers can be obtained
by using performance metrics that measure (1) average fraction of Streaming
Multiprocessors (SMs) used in the device and (2) average fraction of warps used
per SM. Profiling tools such as {\it nvprof} can be used to obtain these
parameters.  

\section{What errors can SASSIFI inject?}
\label{sec:what}

For the D-mode, SASSIFI can inject the error in a destination register based on
the different Bit Flip Models ({\it BFM}).  In the current release (as of
7/23/2016), the following BFMs are implemented. 

\begin{enumerate}
\item Single bit-flip: one bit-flip in one register in one thread
\item Double bit-flip: bit-flips in two adjacent bits in one register in one thread
\item Random value: random value in one register in one thread
\item Zero value: zero out the value of one register in one thread
\item Warp wide single bit-flip: one bit-flip in one register in all the threads in a warp
\item Warp wide double bit-flip: bit-flips in two adjacent bits in one register in all the threads in a warp
\item Warp wide random value: random value in one register in all the threads in a warp
\item Warp wide zero value: zero out the value of one register in all the threads in a warp
\end{enumerate}

In the current implementation, we can only inject single bit-flip in one
register in one thread (first bit-flip model) for the CC and PR injections. For
the SETP-OP instruction group, we can inject only single bit-flip and warp
wide single bit-flip (first and fifth bit-flip models, respectively).

For the I-mode, SASSIFI only considers the following two bit-flip models.  
\begin{itemize}
\item Single bit-flip 
\item Double bit-flip 
\end{itemize}

These BFMs can be extended to include different bit-flip pattern. To add a new
bit-flip model err\_injector.h and injector.cu files in err\_injector directory
and common\_params.py and specific\_params.py files in the scripts directory
need to be modified. 

